%%%%%%%%%%%%%%%%%%%%%%%%%%%%%%%%%%%%%%%%%
% University/School Laboratory Report
% LaTeX Template
% Version 3.1 (25/3/14)
%
% This template has been downloaded from:
% http://www.LaTeXTemplates.com
%
% Original author:
% Linux and Unix Users Group at Virginia Tech Wiki 
% (https://vtluug.org/wiki/Example_LaTeX_chem_lab_report)
%
% License:
% CC BY-NC-SA 3.0 (http://creativecommons.org/licenses/by-nc-sa/3.0/)
%
%%%%%%%%%%%%%%%%%%%%%%%%%%%%%%%%%%%%%%%%%

%----------------------------------------------------------------------------------------
%	PACKAGES AND DOCUMENT CONFIGURATIONS
%----------------------------------------------------------------------------------------

\documentclass{article}

\usepackage[version=3]{mhchem} % Package for chemical equation typesetting
\usepackage{siunitx} % Provides the \SI{}{} and \si{} command for typesetting SI units
\usepackage{graphicx} % Required for the inclusion of images
\usepackage{natbib} % Required to change bibliography style to APA
\usepackage{amsmath} % Required for some math elements 
\usepackage[nodayofweek,level]{datetime}
\usepackage{mdframed}
\usepackage{listings}
\usepackage{enumitem}

\lstset{ %
language=C,                    % choose the language of the code
basicstyle=\footnotesize,         % the size of the fonts that are used for the line-numbers
breaklines=true,
postbreak=\raisebox{0ex}[0ex][0ex]{\ensuremath{\color{red}\hookrightarrow\space}}
}

\setlength\parindent{0pt} % Removes all indentation from paragraphs

%\usepackage{times} % Uncomment to use the Times New Roman font

%----------------------------------------------------------------------------------------
%	DOCUMENT INFORMATION
%----------------------------------------------------------------------------------------

\title{AES Encryption with OpenSSL \\ And Programming RC4 \\ CEEN 8886 --- Wireless Security} % Title

\author{Kelly \textsc{Boswell}} % Author name

\date{\Large February 27, 2017} % Date for the report

\begin{document}

\maketitle % Insert the title, author and date

\begin{center}
\begin{tabular}{l r}
Date Performed: & February 25, 2017 \\ % Date the experiment was performed
Instructor: & Professor Yi Qian % Instructor/supervisor
\end{tabular}
\end{center}

% If you wish to include an abstract, uncomment the lines below
% \begin{abstract}
% Abstract text
% \end{abstract}

%----------------------------------------------------------------------------------------
%	SECTION 1
%----------------------------------------------------------------------------------------

\section{Objective}

The objectives for this lab experiment are as follows:

% If you have more than one objective, uncomment the below:
\begin{itemize}
\item Explore the cryptographic library OpenSSL
\item Conduct AES encryption and decryption with OpenSSL
\item Program RC4 using OpenSSL
\end{itemize}

%----------------------------------------------------------------------------------------
%	SECTION 2
%----------------------------------------------------------------------------------------

\section{Requirements}

Some prerequisites for completing this lab experiment are as follows:

\begin{itemize}
\item OpenSSL cryptographic library
\item A Linux operating system
\item Basic knowledge of the C programming language
\end{itemize}

%----------------------------------------------------------------------------------------
%	SECTION 3
%----------------------------------------------------------------------------------------

\section{Experiment Steps And Results}

\begin{enumerate}
\item \label{step1} Conduct AES encryption and decryption with OpenSSL
    \begin{enumerate}
        \item \label{1a} Create the txt file named ``lab1.txt'' with the line ``Having
               fun with Wireless Security Lab 1'' as your file content. See Figure \ref{fig:step1a}.
        \item \label{1b} Check your content of your file by the cat command. See Figure
               \ref{fig:step1b}. 
        \item \label{1c} Display OpenSSL's help page for encryption options. See Figure \ref{fig:step1c}.
        \item \label{1d} Encrypt ``lab1.txt'' file using AES with CBC mode. Use 256-bit key size 
              with no salting and base64 encoding scheme. Output the file and name it
              ``lab1\textunderscore encrypted.base64''. See Figure \ref{fig:step1d}.
        \item \label{1e} Decrypt ``lab1\textunderscore encrypted.txt” using the same settings as in the previous
              step. Output the file and name it ``lab1\textunderscore decrypted.txt''.
              See Figure \ref{fig:step1e}. Note that the contents of
              ``lab1\textunderscore decrypted.txt'' is exactly the same as the original file,
              ``lab1.txt''.
        \item Repeat Step \ref{1d} --- Step \ref{1e} using the document Lecture02.pdf from the course website.
              The file contents are simply too large to list all of the contents of the encrypted and decrypted
              stages for each of the modes. Instead, the checksums of each of the outputs of the encrypted and
              decrypted stages as well as the original input file are listed in Figure \ref{fig:step1fb}. Figure
              \ref{fig:step1fc} shows the result of `md5sum -c output/checksums.md5sum'. Figure
              \ref{fig:step1fd} shows the execution times of each of the stages. And finally, the shell script
              used to generate all of these results is shown in Figure \ref{fig:step1fa}.

    \end{enumerate}
\end{enumerate}

\begin{figure}
\begin{mdframed}
\begin{lstlisting}
$> echo “Having fun with Wireless Security Lab 1” > lab1.txt
\end{lstlisting}
\end{mdframed}
\caption{Creating the lab1.txt text file.}
\label{fig:step1a}
\end{figure}

\begin{figure}
\begin{mdframed}
\begin{lstlisting}
$> xxd lab1.txt
00000000: 4861 7669 6e67 2066 756e 2077 6974 6820  Having fun with 
00000010: 5769 7265 6c65 7373 2053 6563 7572 6974  Wireless Securit
00000020: 7920 4c61 6220 310a                      y Lab 1.
\end{lstlisting}
\end{mdframed}
\caption{Showing the contents of the lab1.txt text file.}
\label{fig:step1b}
\end{figure}

\begin{figure}
\begin{mdframed}
\begin{lstlisting}
$> openssl enc --help
Usage: enc [options]
Valid options are:
 -help          Display this summary
 -ciphers       List ciphers
 -in infile     Input file
 -out outfile   Output file
 -pass val      Passphrase source
 -e             Encrypt
 -d             Decrypt
 -p             Print the iv/key
 -P             Print the iv/key and exit
 -v             Verbose output
 -nopad         Disable standard block padding
 -salt          Use salt in the KDF (default)
 -nosalt        Do not use salt in the KDF
 -debug         Print debug info
 -a             Base64 encode/decode, depending on encryption
                flag
 -base64        Same as option -a
 -A             Used with -[base64|a] to specify base64 buffer
                as a single line
 -bufsize val   Buffer size
 -k val         Passphrase
 -kfile infile  Read passphrase from file
 -K val         Raw key, in hex
 -S val         Salt, in hex
 -iv val        IV in hex
 -md val        Use specified digest to create a key from the
                passphrase
 -none          Don't encrypt
 -*             Any supported cipher
 -engine val    Use engine, possibly a hardware device
\end{lstlisting}
\end{mdframed}
\caption{Help page OpenSSL encryption.}
\label{fig:step1c}
\end{figure}

\begin{figure}
\begin{mdframed}
\begin{lstlisting}
$> openssl enc --in lab1.txt --out myoutputfile.base64 --base64 --nosalt -e --aes-256-cbc -k "purplemonkeydishwasher"
$> xxd myoutputfile.base64
00000000: 626b 496e 5a76 4a7a 6347 7843 416f 3344  bkInZvJzcGxCAo3D
00000010: 3455 4352 5073 7549 626e 5430 7572 3652  4UCRPsuIbnT0ur6R
00000020: 6c68 7172 6832 7332 5a70 4158 7574 2f50  lhqrh2s2ZpAXut/P
00000030: 7268 646d 6b61 3041 4464 386f 6b50 4246  rhdmka0ADd8okPBF
00000040: 0a
\end{lstlisting}
\end{mdframed}
\caption{Creating and showing the contents of encrypted myoutputfile.base64 text file.}
\label{fig:step1d}
\end{figure}

\begin{figure}
\begin{mdframed}
\begin{lstlisting}
$> openssl enc --in myoutputfile.base64 --out lab1\textunderscore decrypted.txt --base64 --nosalt -d --aes-256-cbc -k "purplemonkeydishwasher"
$> xxd lab1_decrypted.txt
00000000: 4861 7669 6e67 2066 756e 2077 6974 6820  Having fun with 
00000010: 5769 7265 6c65 7373 2053 6563 7572 6974  Wireless Securit
00000020: 7920 4c61 6220 310a                      y Lab 1.
\end{lstlisting}
\end{mdframed}
\caption{Creating and showing the contents of the decrypted lab1\textunderscore decrypted.txt text file.}
\label{fig:step1e}
\end{figure}

\begin{figure}
\begin{mdframed}
\lstinputlisting{enc_dec_all.sh}
\end{mdframed}
\caption{Shell Script for Five Modes of AES Encryption and Decryption.}
\label{fig:step1fa}
\end{figure}

\begin{figure}
\begin{mdframed}
\lstinputlisting{output/checksums.md5sum}
\end{mdframed}
\caption{MD5 Checksums of All AES Encryption and Decryption Inputs and Outputs.}
\label{fig:step1fb}
\end{figure}

\begin{figure}
\begin{mdframed}
\lstinputlisting{output/checksum_result.out}
\end{mdframed}
\caption{MD5 Checksums Result.}
\label{fig:step1fc}
\end{figure}

\begin{figure}
\begin{mdframed}
\lstinputlisting{output/timings.dat}
\end{mdframed}
\caption{AES Encryption Mode and Stage Timings.}
\label{fig:step1fd}
\end{figure}

%----------------------------------------------------------------------------------------
%	SECTION 4
%----------------------------------------------------------------------------------------

\section{Conclusions}

The atomic weight of magnesium is concluded to be \SI{24}{\gram\per\mol}, as determined by the stoichiometry of its chemical combination with oxygen. This result is in agreement with the accepted value.

%----------------------------------------------------------------------------------------
%	SECTION 5
%----------------------------------------------------------------------------------------

\section{Code Listings}

Insert code here.

\end{document}
