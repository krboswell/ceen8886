\documentclass[a4paper]{report}
\usepackage{listings}

\title{CEEN 8886: Homework 2}
\date{2017-02-15}
\author{Kelly Boswell}

\begin{document}

\maketitle

\pagenumbering{gobble}

\newpage

\pagenumbering{arabic}

\section{Problem 1}

Given the 128-bit hexadecimal AES key `2B7E151628AED2A6ABF7158809CF4F3C`, the following key
schedule is generated:

\begin{verbatim}
round = 0 : 54 68 61 74 73 20 6d 79 20 4b 75 6e 67 20 46 75 
round = 1 : e2 32 fc f1 91 12 91 88 b1 59 e4 e6 d6 79 a2 93 
round = 2 : 56 08 20 07 c7 1a b1 8f 76 43 55 69 a0 3a f7 fa 
round = 3 : d2 60 0d e7 15 7a bc 68 63 39 e9 01 c3 03 1e fb 
round = 4 : a1 12 02 c9 b4 68 be a1 d7 51 57 a0 14 52 49 5b 
round = 5 : b1 29 3b 33 05 41 85 92 d2 10 d2 32 c6 42 9b 69 
round = 6 : bd 3d c2 87 b8 7c 47 15 6a 6c 95 27 ac 2e 0e 4e 
round = 7 : cc 96 ed 16 74 ea aa 03 1e 86 3f 24 b2 a8 31 6a 
round = 8 : 8e 51 ef 21 fa bb 45 22 e4 3d 7a 06 56 95 4b 6c 
round = 9 : bf e2 bf 90 45 59 fa b2 a1 64 80 b4 f7 f1 cb d8 
round = 10: 28 fd de f8 6d a4 24 4a cc c0 a4 fe 3b 31 6f 26 
\end{verbatim}

\section{Problem 2}

For the AES algorithm, given the plaintext `2B7E151628AED2A6ABF7158809CF4F3C` and the
128-bit key of all ones, 1) the original contents of \textbf{State} is displayed; 2) the
value of \textbf{State} after the initial AddRoundKey is displayed; 3) the value of \textbf{State}
after SubBytes is displayed; and 4) the value of \textbf{State} after ShiftRows is displayed. However,
since the meaning of `the 128-bit key of all ones` is ambiguous, several different
values of each are displayed along with their corresponding key value in hexadecimal format.

\subsection{Key value of 0x01010101010101010101010101010101} 

Assuming that each byte value of the 128-bit key is 0x01, the output is as follows:

\begin{verbatim}
Initial State:
01 01 01 01 
01 01 01 01 
01 01 01 01 
01 01 01 01 

State after AddRoundKey:
01 05 09 0c 
00 04 08 0d 
03 07 0a 0e 
02 06 0b 0f 

State after SubBytes:
7c 6b 01 fe 
63 f2 30 d7 
7b c5 67 ab 
77 6f 2b 76 

State after ShiftRows:
7c 6b 01 fe 
f2 30 d7 63 
67 ab 7b c5 
76 77 6f 2b 
\end{verbatim}

\subsection{Key value of 0x11111111111111111111111111111111} 

Assuming that each nibble value of the 128-bit key is 0x1, the output is as follows:

\begin{verbatim}
Initial State:
11 11 11 11 
11 11 11 11 
11 11 11 11 
11 11 11 11 

State after AddRoundKey:
11 15 19 1c 
10 14 18 1d 
13 17 1a 1e 
12 16 1b 1f 

State after SubBytes:
82 59 d4 9c 
ca fa ad a4 
7d f0 a2 72 
c9 47 af c0 

State after ShiftRows:
82 59 d4 9c 
fa ad a4 ca 
a2 72 7d f0 
c0 c9 47 af 
\end{verbatim}

\subsection{Key value of 0xFFFFFFFFFFFFFFFFFFFFFFFFFFFFFFFF} 

Assuming that each bit value of the 128-bit key is 1, the output is as follows:

\begin{verbatim}
Initial State:
ff ff ff ff 
ff ff ff ff 
ff ff ff ff 
ff ff ff ff 

State after AddRoundKey:
ff fb f7 f2 
fe fa f6 f3 
fd f9 f4 f0 
fc f8 f5 f1 

State after SubBytes:
16 0f 68 89 
bb 2d 42 0d 
54 99 bf 8c 
b0 41 e6 a1 

State after ShiftRows:
16 0f 68 89 
2d 42 0d bb 
bf 8c 54 99 
a1 b0 41 e6 
\end{verbatim}

\section{Code Listings}

Overall, this code is pretty awfully written. Given more time, I would definitely clean it up quite a bit.
But it did get the job done. It at least gave me correct results when I compared its output with a couple of
examples.

\subsection{hw2.h}

\begin{lstlisting}
#ifndef __HW2_H__
#define __HW2_H__

extern const unsigned char S_BOX[16][16];
extern const unsigned char S_BOX_INV[16][16];
extern const unsigned int RCON[255];

int hex2int(char *achar, int *aint);
int hex2ascii(unsigned char *key);

#endif
\end{lstlisting}

\subsection{hw2.c}

\begin{lstlisting}
#include <stdio.h>

#include "hw2.h"

/*extern*/ const unsigned char S_BOX[16][16] = {
    {0x63, 0x7c, 0x77, 0x7b, 0xf2, 0x6b, 0x6f, 0xc5, 0x30,
        0x01, 0x67, 0x2b, 0xfe, 0xd7, 0xab, 0x76},
    {0xca, 0x82, 0xc9, 0x7d, 0xfa, 0x59, 0x47, 0xf0, 0xad,
        0xd4, 0xa2, 0xaf, 0x9c, 0xa4, 0x72, 0xc0},
    {0xb7, 0xfd, 0x93, 0x26, 0x36, 0x3f, 0xf7, 0xcc, 0x34,
        0xa5, 0xe5, 0xf1, 0x71, 0xd8, 0x31, 0x15},
    {0x04, 0xc7, 0x23, 0xc3, 0x18, 0x96, 0x05, 0x9a, 0x07,
        0x12, 0x80, 0xe2, 0xeb, 0x27, 0xb2, 0x75},
    {0x09, 0x83, 0x2c, 0x1a, 0x1b, 0x6e, 0x5a, 0xa0, 0x52,
        0x3b, 0xd6, 0xb3, 0x29, 0xe3, 0x2f, 0x84},
    {0x53, 0xd1, 0x00, 0xed, 0x20, 0xfc, 0xb1, 0x5b, 0x6a,
        0xcb, 0xbe, 0x39, 0x4a, 0x4c, 0x58, 0xcf},
    {0xd0, 0xef, 0xaa, 0xfb, 0x43, 0x4d, 0x33, 0x85, 0x45,
        0xf9, 0x02, 0x7f, 0x50, 0x3c, 0x9f, 0xa8},
    {0x51, 0xa3, 0x40, 0x8f, 0x92, 0x9d, 0x38, 0xf5, 0xbc,
        0xb6, 0xda, 0x21, 0x10, 0xff, 0xf3, 0xd2},
    {0xcd, 0x0c, 0x13, 0xec, 0x5f, 0x97, 0x44, 0x17, 0xc4,
        0xa7, 0x7e, 0x3d, 0x64, 0x5d, 0x19, 0x73},
    {0x60, 0x81, 0x4f, 0xdc, 0x22, 0x2a, 0x90, 0x88, 0x46,
        0xee, 0xb8, 0x14, 0xde, 0x5e, 0x0b, 0xdb},
    {0xe0, 0x32, 0x3a, 0x0a, 0x49, 0x06, 0x24, 0x5c, 0xc2,
        0xd3, 0xac, 0x62, 0x91, 0x95, 0xe4, 0x79},
    {0xe7, 0xc8, 0x37, 0x6d, 0x8d, 0xd5, 0x4e, 0xa9, 0x6c,
        0x56, 0xf4, 0xea, 0x65, 0x7a, 0xae, 0x08},
    {0xba, 0x78, 0x25, 0x2e, 0x1c, 0xa6, 0xb4, 0xc6, 0xe8,
        0xdd, 0x74, 0x1f, 0x4b, 0xbd, 0x8b, 0x8a},
    {0x70, 0x3e, 0xb5, 0x66, 0x48, 0x03, 0xf6, 0x0e, 0x61,
        0x35, 0x57, 0xb9, 0x86, 0xc1, 0x1d, 0x9e},
    {0xe1, 0xf8, 0x98, 0x11, 0x69, 0xd9, 0x8e, 0x94, 0x9b,
        0x1e, 0x87, 0xe9, 0xce, 0x55, 0x28, 0xdf},
    {0x8c, 0xa1, 0x89, 0x0d, 0xbf, 0xe6, 0x42, 0x68, 0x41,
        0x99, 0x2d, 0x0f, 0xb0, 0x54, 0xbb, 0x16}
};

/*extern*/ const unsigned char S_BOX_INV[16][16] = {
    {0x52, 0x09, 0x6a, 0xd5, 0x30, 0x36, 0xa5, 0x38, 0xbf,
        0x40, 0xa3, 0x9e, 0x81, 0xf3, 0xd7, 0xfb},
    {0x7c, 0xe3, 0x39, 0x82, 0x9b, 0x2f, 0xff, 0x87, 0x34,
        0x8e, 0x43, 0x44, 0xc4, 0xde, 0xe9, 0xcb},
    {0x54, 0x7b, 0x94, 0x32, 0xa6, 0xc2, 0x23, 0x3d, 0xee,
        0x4c, 0x95, 0x0b, 0x42, 0xfa, 0xc3, 0x4e},
    {0x08, 0x2e, 0xa1, 0x66, 0x28, 0xd9, 0x24, 0xb2, 0x76,
        0x5b, 0xa2, 0x49, 0x6d, 0x8b, 0xd1, 0x25},
    {0x72, 0xf8, 0xf6, 0x64, 0x86, 0x68, 0x98, 0x16, 0xd4,
        0xa4, 0x5c, 0xcc, 0x5d, 0x65, 0xb6, 0x92},
    {0x6c, 0x70, 0x48, 0x50, 0xfd, 0xed, 0xb9, 0xda, 0x5e,
        0x15, 0x46, 0x57, 0xa7, 0x8d, 0x9d, 0x84},
    {0x90, 0xd8, 0xab, 0x00, 0x8c, 0xbc, 0xd3, 0x0a, 0xf7,
        0xe4, 0x58, 0x05, 0xb8, 0xb3, 0x45, 0x06},
    {0xd0, 0x2c, 0x1e, 0x8f, 0xca, 0x3f, 0x0f, 0x02, 0xc1,
        0xaf, 0xbd, 0x03, 0x01, 0x13, 0x8a, 0x6b},
    {0x3a, 0x91, 0x11, 0x41, 0x4f, 0x67, 0xdc, 0xea, 0x97,
        0xf2, 0xcf, 0xce, 0xf0, 0xb4, 0xe6, 0x73},
    {0x96, 0xac, 0x74, 0x22, 0xe7, 0xad, 0x35, 0x85, 0xe2,
        0xf9, 0x37, 0xe8, 0x1c, 0x75, 0xdf, 0x6e},
    {0x47, 0xf1, 0x1a, 0x71, 0x1d, 0x29, 0xc5, 0x89, 0x6f,
        0xb7, 0x62, 0x0e, 0xaa, 0x18, 0xbe, 0x1b},
    {0xfc, 0x56, 0x3e, 0x4b, 0xc6, 0xd2, 0x79, 0x20, 0x9a,
        0xdb, 0xc0, 0xfe, 0x78, 0xcd, 0x5a, 0xf4},
    {0x1f, 0xdd, 0xa8, 0x33, 0x88, 0x07, 0xc7, 0x31, 0xb1,
        0x12, 0x10, 0x59, 0x27, 0x80, 0xec, 0x5f},
    {0x60, 0x51, 0x7f, 0xa9, 0x19, 0xb5, 0x4a, 0x0d, 0x2d,
        0xe5, 0x7a, 0x9f, 0x93, 0xc9, 0x9c, 0xef},
    {0xa0, 0xe0, 0x3b, 0x4d, 0xae, 0x2a, 0xf5, 0xb0, 0xc8,
        0xeb, 0xbb, 0x3c, 0x83, 0x53, 0x99, 0x61},
    {0x17, 0x2b, 0x04, 0x7e, 0xba, 0x77, 0xd6, 0x26, 0xe1,
        0x69, 0x14, 0x63, 0x55, 0x21, 0x0c, 0x7d}
};

/*extern*/ const unsigned int RCON[255] = {
    0x8d, 0x01, 0x02, 0x04, 0x08, 0x10, 0x20, 0x40, 0x80,
        0x1b, 0x36, 0x6c, 0xd8, 0xab, 0x4d, 0x9a,
    0x2f, 0x5e, 0xbc, 0x63, 0xc6, 0x97, 0x35, 0x6a, 0xd4,
        0xb3, 0x7d, 0xfa, 0xef, 0xc5, 0x91, 0x39,
    0x72, 0xe4, 0xd3, 0xbd, 0x61, 0xc2, 0x9f, 0x25, 0x4a,
        0x94, 0x33, 0x66, 0xcc, 0x83, 0x1d, 0x3a,
    0x74, 0xe8, 0xcb, 0x8d, 0x01, 0x02, 0x04, 0x08, 0x10,
        0x20, 0x40, 0x80, 0x1b, 0x36, 0x6c, 0xd8,
    0xab, 0x4d, 0x9a, 0x2f, 0x5e, 0xbc, 0x63, 0xc6, 0x97,
        0x35, 0x6a, 0xd4, 0xb3, 0x7d, 0xfa, 0xef,
    0xc5, 0x91, 0x39, 0x72, 0xe4, 0xd3, 0xbd, 0x61, 0xc2,
        0x9f, 0x25, 0x4a, 0x94, 0x33, 0x66, 0xcc,
    0x83, 0x1d, 0x3a, 0x74, 0xe8, 0xcb, 0x8d, 0x01, 0x02,
        0x04, 0x08, 0x10, 0x20, 0x40, 0x80, 0x1b,
    0x36, 0x6c, 0xd8, 0xab, 0x4d, 0x9a, 0x2f, 0x5e, 0xbc,
        0x63, 0xc6, 0x97, 0x35, 0x6a, 0xd4, 0xb3,
    0x7d, 0xfa, 0xef, 0xc5, 0x91, 0x39, 0x72, 0xe4, 0xd3,
        0xbd, 0x61, 0xc2, 0x9f, 0x25, 0x4a, 0x94,
    0x33, 0x66, 0xcc, 0x83, 0x1d, 0x3a, 0x74, 0xe8, 0xcb,
        0x8d, 0x01, 0x02, 0x04, 0x08, 0x10, 0x20,
    0x40, 0x80, 0x1b, 0x36, 0x6c, 0xd8, 0xab, 0x4d, 0x9a,
        0x2f, 0x5e, 0xbc, 0x63, 0xc6, 0x97, 0x35,
    0x6a, 0xd4, 0xb3, 0x7d, 0xfa, 0xef, 0xc5, 0x91, 0x39,
        0x72, 0xe4, 0xd3, 0xbd, 0x61, 0xc2, 0x9f,
    0x25, 0x4a, 0x94, 0x33, 0x66, 0xcc, 0x83, 0x1d, 0x3a,
        0x74, 0xe8, 0xcb, 0x8d, 0x01, 0x02, 0x04,
    0x08, 0x10, 0x20, 0x40, 0x80, 0x1b, 0x36, 0x6c, 0xd8,
        0xab, 0x4d, 0x9a, 0x2f, 0x5e, 0xbc, 0x63,
    0xc6, 0x97, 0x35, 0x6a, 0xd4, 0xb3, 0x7d, 0xfa, 0xef,
        0xc5, 0x91, 0x39, 0x72, 0xe4, 0xd3, 0xbd,
    0x61, 0xc2, 0x9f, 0x25, 0x4a, 0x94, 0x33, 0x66, 0xcc,
        0x83, 0x1d, 0x3a, 0x74, 0xe8, 0xcb
};

int hex2char(char *str, unsigned char *c) {
    int status;
    int digits[2];
    int aint;
    int i;

    status = 0;

    for (i = 0; i < 2; ++i) {
        switch (str[i]) {
            case '0':
                digits[i] = 0;
                break;
            case '1':
                digits[i] = 1;
                break;
            case '2':
                digits[i] = 2;
                break;
            case '3':
                digits[i] = 3;
                break;
            case '4':
                digits[i] = 4;
                break;
            case '5':
                digits[i] = 5;
                break;
            case '6':
                digits[i] = 6;
                break;
            case '7':
                digits[i] = 7;
                break;
            case '8':
                digits[i] = 8;
                break;
            case '9':
                digits[i] = 9;
                break;
            case 'a':
            case 'A':
                digits[i] = 10;
                break;
            case 'b':
            case 'B':
                digits[i] = 11;
                break;
            case 'c':
            case 'C':
                digits[i] = 12;
                break;
            case 'd':
            case 'D':
                digits[i] = 13;
                break;
            case 'e':
            case 'E':
                digits[i] = 14;
                break;
            case 'f':
            case 'F':
                digits[i] = 15;
                break;
            default:
                fprintf(stderr, "Invalid character in hex string: %c\n", str[i]);
                status = 1;
                break;
        }
    }

    if (status == 0) {
        aint = 0;
        aint = (digits[0] << 4) | digits[1];
        *c = (unsigned char) aint;
    }

    return status;
}

int hex2ascii(unsigned char *key) {
    int status;
    char achar[3];
    unsigned char c;
    int i, j;

    status = 0;

    achar[2] = '\0';
    for (i = 0; i < 16; ++i) {
        for (j = 0; j < 2; ++j) {
            achar[j] = key[2 * i + j];
        }
        if (status = hex2char(achar, &c) != 0) {
            break;
        }
        key[i] = c;
    }
    key[16] = '\0';

    return status;
}
\end{lstlisting}

\subsection{pr1.c}

\begin{lstlisting}
#include <ctype.h>
#include <stdio.h>
#include <stdlib.h>
#include <errno.h>
#include <string.h>
#include <unistd.h>
#include <math.h>

#include "hw2.h"

void usage(FILE *out, char *pname);

int copy_key(unsigned char *key, char *orig, int size);
void generate_round_keys(unsigned char *key, unsigned int *w);
void print_round_keys(unsigned int *w);

unsigned int sub_word(unsigned int w);
void sub_bytes(unsigned int *state);
unsigned int sub_byte(unsigned char byte);
unsigned int rot_word(unsigned int w);
void shift_rows(unsigned int *state);

void add_round_key(unsigned int *state, unsigned int *w);
void print_state(unsigned int *state);

int main(int argc, char **argv) {
    int status;
    unsigned char key[33];
    unsigned int w[44];
    unsigned int state[4];

    status = 0;

    if (argc != 3) {
        usage(stderr, argv[0]);
        status = 1;
    }

    if (status == 0) {
        status = copy_key(key, argv[1], 32);
    }

    if (status == 0) {
        status = copy_key((unsigned char *) state, argv[2], 32);
    }

    if (status == 0) {
        status = hex2ascii((unsigned char *) state);
    }

    if (status == 0) {
        status = hex2ascii(key);
    }

    if (status == 0) {
        generate_round_keys(key, w);

        printf("Round keys:\n");
        print_round_keys(w);
        printf("\n");

        printf("Initial State:\n");
        print_state(state);
        printf("\n");

        add_round_key(state, &w[0]);
        printf("State after AddRoundKey:\n");
        print_state(state);
        printf("\n");

        sub_bytes(state);
        printf("State after SubBytes:\n");
        print_state(state);
        printf("\n");

        shift_rows(state);
        printf("State after ShiftRows:\n");
        print_state(state);
        printf("\n");
    }
}

void usage(FILE *out, char *pname) {
    fprintf(out, "Usage: %s <key> <plaintext>\n", pname);
}

int copy_key(unsigned char *key, char *orig, int size) {
    int status;
    size_t s, i;

    status = 0;

    s = strlen(orig);
    if (s != size) {
        fprintf(stderr, "Invalid string length: %lu\n", s);
    } else {
        memcpy(key, orig, sizeof(char) * s);
        key[s] = '\0';
    }

    return status;
}

void generate_round_keys(unsigned char *key, unsigned int *w) {
    int i;
    unsigned int temp, gw, rcon;

    for (i = 0; i < 4; ++i) {
        w[i] = (key[4 * i] << 24)
            | (key[4 * i + 1] << 16)
            | (key[4 * i + 2] << 8)
            | key[4 * i + 3];
    }

    for (i = 4; i < 44; ++i) {
        temp = w[i - 1];
        if ((i % 4) == 0) {
            gw = sub_word(rot_word(temp));
            rcon = RCON[i/4] << 24;
            temp = gw ^ rcon;
        }
        w[i] = w[i - 4] ^ temp;
    }
}

void print_round_keys(unsigned int *w) {
    int i, j, round, index;

    round = 0;
    index = 0;
    for (i = 0; i < 11; ++i) {
        fprintf(stdout, "round = %-2d: ", round);
        for (j = 0; j < 4; ++j) {
            fprintf(stdout, "%02x ", (w[index] & 0xff000000) >> 24);
            fprintf(stdout, "%02x ", (w[index] & 0x00ff0000) >> 16);
            fprintf(stdout, "%02x ", (w[index] & 0x0000ff00) >> 8);
            fprintf(stdout, "%02x ", (w[index] & 0x000000ff));

            ++index;
        }
        fprintf(stdout, "\n");
        ++round;
    }
}

unsigned int sub_word(unsigned int w) {
    unsigned int retval;
    unsigned int row, col;
    unsigned char *cptr;
    int i;

    retval = w;
    cptr = (unsigned char *) &retval;

    for (i = 0; i < 4; ++i) { 
        row = (0xf0 & cptr[i]) >> 4;
        col = (0x0f & cptr[i]);
        cptr[i] = S_BOX[row][col];
    }

    return retval;
}

void sub_bytes(unsigned int *state) {
    int i;
    unsigned char *bytes;

    bytes = (unsigned char *) state;

    for (i = 0; i < 16; ++i) {
        bytes[i] = sub_byte(bytes[i]);
    }
}

unsigned int sub_byte(unsigned char byte) {
    unsigned int row, col;
    row = (0xf0 & byte) >> 4;
    col = (0x0f & byte);
    return S_BOX[row][col];
}

unsigned int rot_word(unsigned int w) {
    unsigned retval;
    unsigned char tmp, *cptr;
    int i;

    retval = w;

    cptr = (unsigned char *) &retval;
    i = 3;

    tmp = cptr[3];
    for (i = 3; i > 0; --i) {
        cptr[i] = cptr[i - 1];
    };
    cptr[0] = tmp;

    return retval;
}

void shift_rows(unsigned int *state) {
    unsigned int tmp, tmp2, tmp3;
    unsigned char *bytes;

    bytes = (unsigned char *) state;
    tmp = bytes[1];
    bytes[1] = bytes[5];
    bytes[5] = bytes[9];
    bytes[9] = bytes[13];
    bytes[13] = tmp;

    tmp = bytes[2];
    tmp2 = bytes[6];
    bytes[2] = bytes[10];
    bytes[6] = bytes[14];
    bytes[10] = tmp;
    bytes[14] = tmp2;

    tmp = bytes[3];
    tmp2 = bytes[7];
    tmp3 = bytes[11];
    bytes[3] = bytes[15];
    bytes[7] = tmp;
    bytes[11] = tmp2;
    bytes[15] = tmp3;
}

void add_round_key(unsigned int *state, unsigned int *w) {
    int i, j;
    unsigned char *state_row;
    unsigned char w_val;

    static unsigned int mask[4] = {
        0xff000000,
        0x00ff0000,
        0x0000ff00,
        0x000000ff
    };

    static unsigned int shift[4] = {
        24,
        16,
        8,
        0
    };

    for (i = 0; i < 4; ++i) {
        state_row = (unsigned char *) &state[i];
        for (j = 0; j < 4; ++j) {
            w_val = (unsigned char)((w[i] & mask[j]) >> shift[j]);
            state_row[j] = state_row[j] ^ w_val;
        }
    }
}

void print_state(unsigned int *state) {
    int i, j;
    unsigned char *bytes;

    bytes = (unsigned char *) state;

    fprintf(stdout, "%02x ", (unsigned int) bytes[0]);
    fprintf(stdout, "%02x ", (unsigned int) bytes[4]);
    fprintf(stdout, "%02x ", (unsigned int) bytes[8]);
    fprintf(stdout, "%02x ", (unsigned int) bytes[12]);
    fprintf(stdout, "\n");
    fprintf(stdout, "%02x ", (unsigned int) bytes[1]);
    fprintf(stdout, "%02x ", (unsigned int) bytes[5]);
    fprintf(stdout, "%02x ", (unsigned int) bytes[9]);
    fprintf(stdout, "%02x ", (unsigned int) bytes[13]);
    fprintf(stdout, "\n");
    fprintf(stdout, "%02x ", (unsigned int) bytes[2]);
    fprintf(stdout, "%02x ", (unsigned int) bytes[6]);
    fprintf(stdout, "%02x ", (unsigned int) bytes[10]);
    fprintf(stdout, "%02x ", (unsigned int) bytes[14]);
    fprintf(stdout, "\n");
    fprintf(stdout, "%02x ", (unsigned int) bytes[3]);
    fprintf(stdout, "%02x ", (unsigned int) bytes[7]);
    fprintf(stdout, "%02x ", (unsigned int) bytes[11]);
    fprintf(stdout, "%02x ", (unsigned int) bytes[15]);
    fprintf(stdout, "\n");
}

\end{lstlisting}

\end{document}
