
%% bare_conf.tex
%% V1.4b
%% 2015/08/26
%% by Michael Shell
%% See:
%% http://www.michaelshell.org/
%% for current contact information.
%%
%% This is a skeleton file demonstrating the use of IEEEtran.cls
%% (requires IEEEtran.cls version 1.8b or later) with an IEEE
%% conference paper.
%%
%% Support sites:
%% http://www.michaelshell.org/tex/ieeetran/
%% http://www.ctan.org/pkg/ieeetran
%% and
%% http://www.ieee.org/

%%*************************************************************************
%% Legal Notice:
%% This code is offered as-is without any warranty either expressed or
%% implied; without even the implied warranty of MERCHANTABILITY or
%% FITNESS FOR A PARTICULAR PURPOSE! 
%% User assumes all risk.
%% In no event shall the IEEE or any contributor to this code be liable for
%% any damages or losses, including, but not limited to, incidental,
%% consequential, or any other damages, resulting from the use or misuse
%% of any information contained here.
%%
%% All comments are the opinions of their respective authors and are not
%% necessarily endorsed by the IEEE.
%%
%% This work is distributed under the LaTeX Project Public License (LPPL)
%% ( http://www.latex-project.org/ ) version 1.3, and may be freely used,
%% distributed and modified. A copy of the LPPL, version 1.3, is included
%% in the base LaTeX documentation of all distributions of LaTeX released
%% 2003/12/01 or later.
%% Retain all contribution notices and credits.
%% ** Modified files should be clearly indicated as such, including  **
%% ** renaming them and changing author support contact information. **
%%*************************************************************************


% *** Authors should verify (and, if needed, correct) their LaTeX system  ***
% *** with the testflow diagnostic prior to trusting their LaTeX platform ***
% *** with production work. The IEEE's font choices and paper sizes can   ***
% *** trigger bugs that do not appear when using other class files.       ***                          ***
% The testflow support page is at:
% http://www.michaelshell.org/tex/testflow/



\documentclass[10pt,conference]{IEEEtran}
\usepackage{listings}
\usepackage{graphicx}
\usepackage{mdframed}
\usepackage{grffile}
% Some Computer Society conferences also require the compsoc mode option,
% but others use the standard conference format.
%
% If IEEEtran.cls has not been installed into the LaTeX system files,
% manually specify the path to it like:
% \documentclass[conference]{../sty/IEEEtran}





% Some very useful LaTeX packages include:
% (uncomment the ones you want to load)


% *** MISC UTILITY PACKAGES ***
%
%\usepackage{ifpdf}
% Heiko Oberdiek's ifpdf.sty is very useful if you need conditional
% compilation based on whether the output is pdf or dvi.
% usage:
% \ifpdf
%   % pdf code
% \else
%   % dvi code
% \fi
% The latest version of ifpdf.sty can be obtained from:
% http://www.ctan.org/pkg/ifpdf
% Also, note that IEEEtran.cls V1.7 and later provides a builtin
% \ifCLASSINFOpdf conditional that works the same way.
% When switching from latex to pdflatex and vice-versa, the compiler may
% have to be run twice to clear warning/error messages.






% *** CITATION PACKAGES ***
%
\usepackage{cite}
% cite.sty was written by Donald Arseneau
% V1.6 and later of IEEEtran pre-defines the format of the cite.sty package
% \cite{} output to follow that of the IEEE. Loading the cite package will
% result in citation numbers being automatically sorted and properly
% "compressed/ranged". e.g., [1], [9], [2], [7], [5], [6] without using
% cite.sty will become [1], [2], [5]--[7], [9] using cite.sty. cite.sty's
% \cite will automatically add leading space, if needed. Use cite.sty's
% noadjust option (cite.sty V3.8 and later) if you want to turn this off
% such as if a citation ever needs to be enclosed in parenthesis.
% cite.sty is already installed on most LaTeX systems. Be sure and use
% version 5.0 (2009-03-20) and later if using hyperref.sty.
% The latest version can be obtained at:
% http://www.ctan.org/pkg/cite
% The documentation is contained in the cite.sty file itself.






% *** GRAPHICS RELATED PACKAGES ***
%
\ifCLASSINFOpdf
  % \usepackage[pdftex]{graphicx}
  % declare the path(s) where your graphic files are
  % \graphicspath{{../pdf/}{../jpeg/}}
  % and their extensions so you won't have to specify these with
  % every instance of \includegraphics
  % \DeclareGraphicsExtensions{.pdf,.jpeg,.png}
\else
  % or other class option (dvipsone, dvipdf, if not using dvips). graphicx
  % will default to the driver specified in the system graphics.cfg if no
  % driver is specified.
  % \usepackage[dvips]{graphicx}
  % declare the path(s) where your graphic files are
  % \graphicspath{{../eps/}}
  % and their extensions so you won't have to specify these with
  % every instance of \includegraphics
  % \DeclareGraphicsExtensions{.eps}
\fi
% graphicx was written by David Carlisle and Sebastian Rahtz. It is
% required if you want graphics, photos, etc. graphicx.sty is already
% installed on most LaTeX systems. The latest version and documentation
% can be obtained at: 
% http://www.ctan.org/pkg/graphicx
% Another good source of documentation is "Using Imported Graphics in
% LaTeX2e" by Keith Reckdahl which can be found at:
% http://www.ctan.org/pkg/epslatex
%
% latex, and pdflatex in dvi mode, support graphics in encapsulated
% postscript (.eps) format. pdflatex in pdf mode supports graphics
% in .pdf, .jpeg, .png and .mps (metapost) formats. Users should ensure
% that all non-photo figures use a vector format (.eps, .pdf, .mps) and
% not a bitmapped formats (.jpeg, .png). The IEEE frowns on bitmapped formats
% which can result in "jaggedy"/blurry rendering of lines and letters as
% well as large increases in file sizes.
%
% You can find documentation about the pdfTeX application at:
% http://www.tug.org/applications/pdftex





% *** MATH PACKAGES ***
%
%\usepackage{amsmath}
% A popular package from the American Mathematical Society that provides
% many useful and powerful commands for dealing with mathematics.
%
% Note that the amsmath package sets \interdisplaylinepenalty to 10000
% thus preventing page breaks from occurring within multiline equations. Use:
%\interdisplaylinepenalty=2500
% after loading amsmath to restore such page breaks as IEEEtran.cls normally
% does. amsmath.sty is already installed on most LaTeX systems. The latest
% version and documentation can be obtained at:
% http://www.ctan.org/pkg/amsmath





% *** SPECIALIZED LIST PACKAGES ***
%
%\usepackage{algorithmic}
% algorithmic.sty was written by Peter Williams and Rogerio Brito.
% This package provides an algorithmic environment fo describing algorithms.
% You can use the algorithmic environment in-text or within a figure
% environment to provide for a floating algorithm. Do NOT use the algorithm
% floating environment provided by algorithm.sty (by the same authors) or
% algorithm2e.sty (by Christophe Fiorio) as the IEEE does not use dedicated
% algorithm float types and packages that provide these will not provide
% correct IEEE style captions. The latest version and documentation of
% algorithmic.sty can be obtained at:
% http://www.ctan.org/pkg/algorithms
% Also of interest may be the (relatively newer and more customizable)
% algorithmicx.sty package by Szasz Janos:
% http://www.ctan.org/pkg/algorithmicx




% *** ALIGNMENT PACKAGES ***
%
%\usepackage{array}
% Frank Mittelbach's and David Carlisle's array.sty patches and improves
% the standard LaTeX2e array and tabular environments to provide better
% appearance and additional user controls. As the default LaTeX2e table
% generation code is lacking to the point of almost being broken with
% respect to the quality of the end results, all users are strongly
% advised to use an enhanced (at the very least that provided by array.sty)
% set of table tools. array.sty is already installed on most systems. The
% latest version and documentation can be obtained at:
% http://www.ctan.org/pkg/array


% IEEEtran contains the IEEEeqnarray family of commands that can be used to
% generate multiline equations as well as matrices, tables, etc., of high
% quality.




% *** SUBFIGURE PACKAGES ***
%\ifCLASSOPTIONcompsoc
%  \usepackage[caption=false,font=normalsize,labelfont=sf,textfont=sf]{subfig}
%\else
%  \usepackage[caption=false,font=footnotesize]{subfig}
%\fi
% subfig.sty, written by Steven Douglas Cochran, is the modern replacement
% for subfigure.sty, the latter of which is no longer maintained and is
% incompatible with some LaTeX packages including fixltx2e. However,
% subfig.sty requires and automatically loads Axel Sommerfeldt's caption.sty
% which will override IEEEtran.cls' handling of captions and this will result
% in non-IEEE style figure/table captions. To prevent this problem, be sure
% and invoke subfig.sty's "caption=false" package option (available since
% subfig.sty version 1.3, 2005/06/28) as this is will preserve IEEEtran.cls
% handling of captions.
% Note that the Computer Society format requires a larger sans serif font
% than the serif footnote size font used in traditional IEEE formatting
% and thus the need to invoke different subfig.sty package options depending
% on whether compsoc mode has been enabled.
%
% The latest version and documentation of subfig.sty can be obtained at:
% http://www.ctan.org/pkg/subfig




% *** FLOAT PACKAGES ***
%
%\usepackage{fixltx2e}
% fixltx2e, the successor to the earlier fix2col.sty, was written by
% Frank Mittelbach and David Carlisle. This package corrects a few problems
% in the LaTeX2e kernel, the most notable of which is that in current
% LaTeX2e releases, the ordering of single and double column floats is not
% guaranteed to be preserved. Thus, an unpatched LaTeX2e can allow a
% single column figure to be placed prior to an earlier double column
% figure.
% Be aware that LaTeX2e kernels dated 2015 and later have fixltx2e.sty's
% corrections already built into the system in which case a warning will
% be issued if an attempt is made to load fixltx2e.sty as it is no longer
% needed.
% The latest version and documentation can be found at:
% http://www.ctan.org/pkg/fixltx2e


%\usepackage{stfloats}
% stfloats.sty was written by Sigitas Tolusis. This package gives LaTeX2e
% the ability to do double column floats at the bottom of the page as well
% as the top. (e.g., "\begin{figure*}[!b]" is not normally possible in
% LaTeX2e). It also provides a command:
%\fnbelowfloat
% to enable the placement of footnotes below bottom floats (the standard
% LaTeX2e kernel puts them above bottom floats). This is an invasive package
% which rewrites many portions of the LaTeX2e float routines. It may not work
% with other packages that modify the LaTeX2e float routines. The latest
% version and documentation can be obtained at:
% http://www.ctan.org/pkg/stfloats
% Do not use the stfloats baselinefloat ability as the IEEE does not allow
% \baselineskip to stretch. Authors submitting work to the IEEE should note
% that the IEEE rarely uses double column equations and that authors should try
% to avoid such use. Do not be tempted to use the cuted.sty or midfloat.sty
% packages (also by Sigitas Tolusis) as the IEEE does not format its papers in
% such ways.
% Do not attempt to use stfloats with fixltx2e as they are incompatible.
% Instead, use Morten Hogholm'a dblfloatfix which combines the features
% of both fixltx2e and stfloats:
%
% \usepackage{dblfloatfix}
% The latest version can be found at:
% http://www.ctan.org/pkg/dblfloatfix




% *** PDF, URL AND HYPERLINK PACKAGES ***
%
%\usepackage{url}
% url.sty was written by Donald Arseneau. It provides better support for
% handling and breaking URLs. url.sty is already installed on most LaTeX
% systems. The latest version and documentation can be obtained at:
% http://www.ctan.org/pkg/url
% Basically, \url{my_url_here}.




% *** Do not adjust lengths that control margins, column widths, etc. ***
% *** Do not use packages that alter fonts (such as pslatex).         ***
% There should be no need to do such things with IEEEtran.cls V1.6 and later.
% (Unless specifically asked to do so by the journal or conference you plan
% to submit to, of course. )

\lstset{
language=C,                    % choose the language of the code
basicstyle=\footnotesize,         % the size of the fonts that are used for the line-numbers
breaklines=true,
postbreak=\raisebox{0ex}[0ex][0ex]{\ensuremath{\color{red}\hookrightarrow\space}}
}

% correct bad hyphenation here
\hyphenation{op-tical net-works semi-conduc-tor}


\begin{document}
%
% paper title
% Titles are generally capitalized except for words such as a, an, and, as,
% at, but, by, for, in, nor, of, on, or, the, to and up, which are usually
% not capitalized unless they are the first or last word of the title.
% Linebreaks \\ can be used within to get better formatting as desired.
% Do not put math or special symbols in the title.
\title{CEEN 8886 Course Project\\ Network Security Evaluation}


% author names and affiliations
% use a multiple column layout for up to three different
% affiliations
\author{\IEEEauthorblockN{Kelly Boswell}
\IEEEauthorblockA{School of Electrical and\\Computer Engineering\\
University of Nebraska --- Lincoln\\
Lincoln, NE 68588\\
Email: krboswell@gmail.com}}

% conference papers do not typically use \thanks and this command
% is locked out in conference mode. If really needed, such as for
% the acknowledgment of grants, issue a \IEEEoverridecommandlockouts
% after \documentclass

% for over three affiliations, or if they all won't fit within the width
% of the page, use this alternative format:
% 
%\author{\IEEEauthorblockN{Michael Shell\IEEEauthorrefmark{1},
%Homer Simpson\IEEEauthorrefmark{2},
%James Kirk\IEEEauthorrefmark{3}, 
%Montgomery Scott\IEEEauthorrefmark{3} and
%Eldon Tyrell\IEEEauthorrefmark{4}}
%\IEEEauthorblockA{\IEEEauthorrefmark{1}School of Electrical and Computer Engineering\\
%Georgia Institute of Technology,
%Atlanta, Georgia 30332--0250\\ Email: see http://www.michaelshell.org/contact.html}
%\IEEEauthorblockA{\IEEEauthorrefmark{2}Twentieth Century Fox, Springfield, USA\\
%Email: homer@thesimpsons.com}
%\IEEEauthorblockA{\IEEEauthorrefmark{3}Starfleet Academy, San Francisco, California 96678-2391\\
%Telephone: (800) 555--1212, Fax: (888) 555--1212}
%\IEEEauthorblockA{\IEEEauthorrefmark{4}Tyrell Inc., 123 Replicant Street, Los Angeles, California 90210--4321}}




% use for special paper notices
%\IEEEspecialpapernotice{(Invited Paper)}




% make the title area
\maketitle

% As a general rule, do not put math, special symbols or citations
% in the abstract
\begin{abstract}
The purpose of this project is to learn how to use some networking tools for network security assessment. There
are two parts to this project. In the first part, the EICAR virus is used to to test a number of scenarios for distributing
viruses through email. In this part, it is found that viruses are spread very easily via the SMTP/MIME and POP protocols and that
the anti-virus software is either ineffective in certain scenarios or that the anti-virus software doesn't scan certain fields
because plaintext viruses are known to be innocuous in those fields. In the second part, security of SSL/TLS over HTTP is evaluated.
In this part of the project, it is again found that viruses spread very easily via HTTP assuming that a website is hosting viruses
either intentionally or unintentionally. In all scenarios tested, the anti-virus software detects the virus. It is also found that
SSL/TLS provides adequate privacy for HTTP session unless the server's private key is compromised. Then a network packet capture
with Wireshark reveals all contents just as if it were a non-secure session. 
\end{abstract}

% no keywords
\begin{IEEEkeywords}
Network security, Wireshark, email, virus, anti-virus, EICAR, SMTP, POP, HTTP, SSL, TLS
\end{IEEEkeywords}



% For peer review papers, you can put extra information on the cover
% page as needed:
% \ifCLASSOPTIONpeerreview
% \begin{center} \bfseries EDICS Category: 3-BBND \end{center}
% \fi
%
% For peerreview papers, this IEEEtran command inserts a page break and
% creates the second title. It will be ignored for other modes.
\IEEEpeerreviewmaketitle



\section{Introduction}
% no \IEEEPARstart
The purpose of this project is to learn how to use some networking tools for network security assessment. First, the basic setup
of the project is discussed. Next, the EICAR virus is used to to test a number of scenarios for distributing
viruses through email. Anti-virus software\footnote{The anti-virus software used throughout this project is Symantec Endpoint Protection.}
is used to test the effectiveness of the software when distributing the virus
via different fields in the SMTP protocol.  Some of the protocols explored in this part are SMTP, POP, and MIME. Lastly,
security of SSL/TLS over HTTP is evaluated. Some of the protocols explored in this part are SSL/TLS, and HTTP.

% An example of a floating figure using the graphicx package.
% Note that \label must occur AFTER (or within) \caption.
% For figures, \caption should occur after the \includegraphics.
% Note that IEEEtran v1.7 and later has special internal code that
% is designed to preserve the operation of \label within \caption
% even when the captionsoff option is in effect. However, because
% of issues like this, it may be the safest practice to put all your
% \label just after \caption rather than within \caption{}.
%
% Reminder: the "draftcls" or "draftclsnofoot", not "draft", class
% option should be used if it is desired that the figures are to be
% displayed while in draft mode.
%
%\begin{figure}[!t]
%\centering
%\includegraphics[width=2.5in]{myfigure}
% where an .eps filename suffix will be assumed under latex, 
% and a .pdf suffix will be assumed for pdflatex; or what has been declared
% via \DeclareGraphicsExtensions.
%\caption{Simulation results for the network.}
%\label{fig_sim}
%\end{figure}

% Note that the IEEE typically puts floats only at the top, even when this
% results in a large percentage of a column being occupied by floats.


% An example of a double column floating figure using two subfigures.
% (The subfig.sty package must be loaded for this to work.)
% The subfigure \label commands are set within each subfloat command,
% and the \label for the overall figure must come after \caption.
% \hfil is used as a separator to get equal spacing.
% Watch out that the combined width of all the subfigures on a 
% line do not exceed the text width or a line break will occur.
%
%\begin{figure*}[!t]
%\centering
%\subfloat[Case I]{\includegraphics[width=2.5in]{box}%
%\label{fig_first_case}}
%\hfil
%\subfloat[Case II]{\includegraphics[width=2.5in]{box}%
%\label{fig_second_case}}
%\caption{Simulation results for the network.}
%\label{fig_sim}
%\end{figure*}
%
% Note that often IEEE papers with subfigures do not employ subfigure
% captions (using the optional argument to \subfloat[]), but instead will
% reference/describe all of them (a), (b), etc., within the main caption.
% Be aware that for subfig.sty to generate the (a), (b), etc., subfigure
% labels, the optional argument to \subfloat must be present. If a
% subcaption is not desired, just leave its contents blank,
% e.g., \subfloat[].


% An example of a floating table. Note that, for IEEE style tables, the
% \caption command should come BEFORE the table and, given that table
% captions serve much like titles, are usually capitalized except for words
% such as a, an, and, as, at, but, by, for, in, nor, of, on, or, the, to
% and up, which are usually not capitalized unless they are the first or
% last word of the caption. Table text will default to \footnotesize as
% the IEEE normally uses this smaller font for tables.
% The \label must come after \caption as always.
%
%\begin{table}[!t]
%% increase table row spacing, adjust to taste
%\renewcommand{\arraystretch}{1.3}
% if using array.sty, it might be a good idea to tweak the value of
% \extrarowheight as needed to properly center the text within the cells
%\caption{An Example of a Table}
%\label{table_example}
%\centering
%% Some packages, such as MDW tools, offer better commands for making tables
%% than the plain LaTeX2e tabular which is used here.
%\begin{tabular}{|c||c|}
%\hline
%One & Two\\
%\hline
%Three & Four\\
%\hline
%\end{tabular}
%\end{table}


% Note that the IEEE does not put floats in the very first column
% - or typically anywhere on the first page for that matter. Also,
% in-text middle ("here") positioning is typically not used, but it
% is allowed and encouraged for Computer Society conferences (but
% not Computer Society journals). Most IEEE journals/conferences use
% top floats exclusively. 
% Note that, LaTeX2e, unlike IEEE journals/conferences, places
% footnotes above bottom floats. This can be corrected via the
% \fnbelowfloat command of the stfloats package.




\section{Setup}
The test apparatus consists of two virtual machines: one used as an email and HTTP
server and the other used as the email and HTTP client.  The server platform consists of the
CentOS 7 operating system using postfix for basic SMTP service, dovecot for IMAP/POP3
service, and httpd for HTTP service. The client platform consists of the Windows 7 operating
system, the Thunderbird email client, and the Internet Explorer web browser. Both platforms
use Symantec Endpoint Protection for anti-virus service.

On the server node, postfix, dovecot, and httpd may be installed via the yum repositories. It may
be necessary to remove sendmail, first. The server is given a static IP address on a private network
by modifying the appropriate interface script in $/etc/sysconfig/network-scripts$ and its fully-qualified
hostname is set to $mail01.acme.com$. To simplify the experiment, SELINUX and the firewall are 
disabled. Then, some postfix and dovecot configuration files must be updated\cite{centos:email:sk} 
in order for the programs to interoperate correctly and to make the service accessible to local clients.
Lastly, a private key and a public key must be created and the httpd configurations must be updated to 
reference them in order to serve HTTP over SSL/TLS\cite{centos:http:cg}.

On the client node, Thunderbird\cite{windows:thunderbird} and Wireshark\cite{windows:wireshark} may
be downloaded and installed from their host websites.
And, source code for netcat for Windows\cite{windows:netcat:jc} must be downloaded and compiled
with MinGW\cite{windows:mingw} C compilers.

\section{Email Service Security Evaluation}
The EICAR virus\cite{windows:eicar} test string is used in a number of SMTP
and MIME fields to test the dependability of anti-virus software on both the server and host platforms.
The test scenarios include: inserting the string into the message body (Figure \ref{fig:body}); inserting the string into the subject
(Figure \ref{fig:subject});
inserting the string into an HTML table (Figure \ref{fig:html}); inserting the string into the comments (Figure \ref{fig:comments});  inserting the string into the
keywords (Figure \ref{fig:keywords}); inserting the string into the references (Figure \ref{fig:references}); inserting the test string in a text file attachment
(Figure \ref{fig:attachment}).
The messages were retrieved both via a netcat script with a POP retrieval messages and via the Thunderbird mail client.
In each case, only the emails with the test string in the message body and with the test string an an attached file triggered
alerts from the anti-virus program.

Wireshark is also used to monitor network traffic on the client while the message with the attachment (Figure \ref{fig:attachment}) is retrieved. There is no discernable
difference between the retrieved message using netcat and the retrieved message using the Thunderbird client except that different session ID's 
are used for postfix and a different Message-Id is used for each.

Surprisingly, some of the other tests didn't trigger alerts. This may be because
the anti-virus program doesn't look in these fields for potential virus strings or that virus strings are simply innocuous in them.
In the case of the HTML table, it's likely that the test virus string is innocuous when inserted into an HTML body as it is.

\begin{figure}
\begin{mdframed}
\lstinputlisting{smtp/smtp-mime-body.nc}
\caption{Inserting EICAR string to the message body.}
\label{fig:body}
\end{mdframed}
\end{figure}

\begin{figure}
\begin{mdframed}
\lstinputlisting{smtp/smtp-mime-subject.nc}
\caption{Inserting EICAR string to the subject}
\label{fig:subject}
\end{mdframed}
\end{figure}

\begin{figure}
\begin{mdframed}
\lstinputlisting{smtp/smtp-mime-comments.nc}
\caption{Inserting EICAR string to the comments.}
\label{fig:comments}
\end{mdframed}
\end{figure}

\begin{figure}
\begin{mdframed}
\lstinputlisting{smtp/smtp-mime-keywords.nc}
\caption{Inserting EICAR string to the keywords.}
\label{fig:keywords}
\end{mdframed}
\end{figure}

\begin{figure}
\begin{mdframed}
\lstinputlisting{smtp/smtp-mime-references.nc}
\caption{Inserting EICAR string to the references.}
\label{fig:references}
\end{mdframed}
\end{figure}

\begin{figure}
\begin{mdframed}
\lstinputlisting{smtp/smtp-mime-html.nc}
\caption{Inserting EICAR string in HTML table.}
\label{fig:html}
\end{mdframed}
\end{figure}

\begin{figure}
\begin{mdframed}
\lstinputlisting{smtp/smtp-mime.nc}
\caption{Inserting EICAR string file as an attachment.}
\label{fig:attachment}
\end{mdframed}
\end{figure}

\section{HTTP Service Security Evaluation}
The EICAR virus test string is copied into a number of files with different suffixes: eicar.com, 
eicar.txt, eicar.html, and eicar.gif. These files are then copied into the web content folder so
that they may be retrieved over HTTP and HTTPS. When doing so, the server's anti-virus 
program generates an alert until the web content directory is whitelisted. From the client, the files may be retrieved
using netcat (Figure \ref{fig:httpnetcat}) or a web browser.  Either way, an alert is generated on the client in all cases, despite
the file suffix.  This is an expected outcome.

The process is repeated but this time over HTTPS so the communication is secured with SSL/TLS.
Wireshark's SSL decryption capability is used here to simplify the process of decrypting all of
the captured network traffic on the client. In order to do so, it is imperative that RSA is used
as the encryption scheme on the server. On CentOS 7, it may be necessary to modify the httpd configuration
file $ssl.conf$'s $SSLCipherSuite$ configuration value to $RSA+AESGCM:RSA+AES:RSA+3DES:!aNULL:!eNULL:!LOW:!3DES:!MD5:!EXP:!PSK:!SRP:!DSS$.
Otherwise, httpd on CentOS 7 seems to prefer Diffie-Hellman over RSA even if an RSA private and public
key are generated and configured for use. Then, the private key used by the HTTP server must be copied 
to the client and Wireshark must be configured so that the SSL protocol will decrypt HTTP traffic with it.

If the private key is accessible, then it is possible to capture and decrypt all network traffic, including the entire key exchange protocol.
The HTTP traffic may include the server's operating system and web server application and version number. So, not only
is privacy for web traffic for that server compromised, but some critical information may be leaked to nefarious parties, 
assuming encryption is intended to keep such information private.

\begin{figure}
\begin{mdframed}
\lstinputlisting{http/http.nc}
\caption{Retrieving content from the web server using netcat.}
\label{fig:httpnetcat}
\end{mdframed}
\end{figure}

\section{Conclusion}
This project explores some applications commonly used in a networked environment and the security issues
surrounding them. First, an email server using the SMTP/MIME and POP protocols are used to deliver a test virus between
a client and server to test the anti-virus software's ability to detect it. While doing so, it was discovered that the software
wasn't able to detect the virus it's embedded in some of the SMTP/MIME fields but it's possible that the test virus string 
is innocuous when it's embedded in those fields. It may also be that the anti-virus scanner doesn't look at those fields
and it is a potential vulnerability. This shows how easily viruses may spread via email.

Next, the potential for delivering the same virus via an HTTP service is evaluated. If the virus is somehow
placed in a file on the web server, it could be distributed to any clients unlucky enough to view it. It could also be that the
website is hosted by a nefarious party intent on distributing the virus. It was discovered that the anti-virus software was
able to detect the test virus despite the file suffix in all cases, however. Even still, this finding shows how easily viruses
may be distributed over the world wide web.  

Lastly, the security of SSL/TLS on HTTP is evaluated. Wireshark was used to decrypt an entire SSL/TLS session but only
when the server's private key was known to the client using Wireshark and that's only possible if an insider accidentally
or purposefully leaks the server's private key or if the server is penetrated by a nefarious party. Otherwise, SSL/TLS provides adequate privacy
for clients using the server.  

% conference papers do not normally have an appendix


% use section* for acknowledgment
%\section*{Acknowledgment}


%The authors would like to thank...





% trigger a \newpage just before the given reference
% number - used to balance the columns on the last page
% adjust value as needed - may need to be readjusted if
% the document is modified later
%\IEEEtriggeratref{8}
% The "triggered" command can be changed if desired:
%\IEEEtriggercmd{\enlargethispage{-5in}}

% references section

% can use a bibliography generated by BibTeX as a .bbl file
% BibTeX documentation can be easily obtained at:
% http://mirror.ctan.org/biblio/bibtex/contrib/doc/
% The IEEEtran BibTeX style support page is at:
% http://www.michaelshell.org/tex/ieeetran/bibtex/
%\bibliographystyle{IEEEtran}
% argument is your BibTeX string definitions and bibliography database(s)
%\bibliography{IEEEabrv,../bib/paper}
%
% <OR> manually copy in the resultant .bbl file
% set second argument of \begin to the number of references
% (used to reserve space for the reference number labels box)
\begin{thebibliography}{6}

\bibitem{centos:email:sk}
Senthil Kumar.  (2016) \emph{Set Up a Local Mail Server in CentOS 7} [Online].
  Available: https://www.unixmen.com/setup-a-local-mail-server-in-centos-7/
  
\bibitem{centos:http:cg}
Christoph Galuschka.  (2014) \emph{Setting up an SSL secured Webserver with CentOS} [Online].
  Available: https://wiki.centos.org/HowTos/Https

\bibitem{windows:netcat:jc}
Jon Craton.  (2009) \emph{Netcat for Windows} [Online].
  Available: https://joncraton.org/blog/46/netcat-for-windows/

\bibitem{windows:thunderbird}
(2017) \emph{Thunderbird} [Online].
  Available: https://www.mozilla.org/en-US/thunderbird/

\bibitem{windows:wireshark}
(2017) \emph{Wireshark} [Online].
  Available: https://www.wireshark.org/

\bibitem{windows:mingw}
(2017) \emph{MinGW --- Minimalist GNU for Windows} [Online].
  Available: http://www.mingw.org/

\bibitem{windows:eicar}
(2017) \emph{Anti-Malware Test File --- Intended Use} [Online].
  Available: http://www.eicar.org/86-0-Intended-use.html

\end{thebibliography}


% that's all folks
\end{document}


\grid
